% -*- Mode:TeX -*-

%% IMPORTANT: The official thesis specifications are available at:
%%            http://libraries.mit.edu/archives/thesis-specs/
%%
%%            Please verify your thesis' formatting and copyright
%%            assignment before submission.  If you notice any
%%            discrepancies between these templates and the 
%%            MIT Libraries' specs, please let us know
%%            by e-mailing thesis@mit.edu

%% The documentclass options along with the pagestyle can be used to generate
%% a technical report, a draft copy, or a regular thesis.  You may need to
%% re-specify the pagestyle after you \include  cover.tex.  For more
%% information, see the first few lines of mitthesis.cls. 

%\documentclass[12pt,vi,twoside]{mitthesis}
%%
%%  If you want your thesis copyright to you instead of MIT, use the
%%  ``vi'' option, as above.
%%
%\documentclass[12pt,twoside,leftblank]{mitthesis}
%%
%% If you want blank pages before new chapters to be labelled ``This
%% Page Intentionally Left Blank'', use the ``leftblank'' option, as
%% above. 

\documentclass[12pt,twoside]{mitthesis}
%\nofiles

\PassOptionsToPackage{table,xcdraw}{xcolor}


\usepackage{lgrind}
%% These have been added at the request of the MIT Libraries, because
%% some PDF conversions mess up the ligatures.  -LB, 1/22/2014
\usepackage{cmap}
\usepackage[T1]{fontenc}
%% Package "lmodern" added by user request see ServiceNow INC0396734 -OT, 4/29/2020
\usepackage{lmodern}

\usepackage[center]{caption}
\usepackage[utf8]{inputenc}
\usepackage{amsmath}
\usepackage{amsfonts}
\usepackage[a4paper, total={7in, 9.5in}]{geometry}
\usepackage{graphicx}
\usepackage[section]{placeins}
%\usepackage{cite}
\usepackage{amsmath,amssymb,amsfonts}
\usepackage{graphicx}
\usepackage{textcomp}
\usepackage{xcolor}
\usepackage{changepage}
\usepackage{hyperref}
\usepackage{nomencl}
%\usepackage{unicode-math}
\usepackage{amsthm}
\usepackage{enumitem}  
\usepackage{calrsfs}
\usepackage{dutchcal}


\usepackage{tabularx,colortbl}

%\usepackage{colortbl}
%\usepackage{xcolor}

\DeclareMathAlphabet{\pazocal}{OMS}{zplm}{m}{n}
\SetMathAlphabet\pazocal{bold}{OMS}{zplm}{bx}{n}
\newcommand{\mth}{\mathcal{A}}
\newcommand{\pzc}{\pazocal{A}}

\usepackage{tikz}
\newcommand*\circled[1]{\tikz[baseline=(char.base)]{
            \node[shape=circle,draw,inner sep=2pt] (char) {#1};}}
\newcommand\norm[1]{\left\lVert#1\right\rVert}
\usetikzlibrary{calc,shapes.misc,shapes.geometric}
\usepackage{array,booktabs}




\newcommand\Mark[2][]{%
\tikz[baseline=(a.base)]{
\node[inner sep=0pt,outer sep=0pt](a){\phantom{#2}};  %% just to be defensive
\node[draw,thick,inner sep=1pt, ellipse,text=black,overlay,minimum width=\widthof{#2},#1]  {#2};%
}
}

\newcommand{\colorcell}{\textcolor{red}}

\makeatletter
\newcommand{\repeatable}[2]{%
  \label{#1}\global\@namedef{repeatable@#1}{#2}#2
}
\newcommand{\eqrepeat}[1]{%
  \@ifundefined{repeatable@#1}{NOT FOUND}{$\@nameuse{repeatable@#1}$}%
  ~\eqref{#1}}
\makeatother


\newtheorem{theorem}{Theorem}[section]
\newtheorem{corollary}{Corollary}[theorem]
\newtheorem{lemma}[theorem]{Lemma}
\newtheorem{assumption}{Assumption}

\newtheorem{definition}{Definition}[section]
\newtheorem{prop}{Proposition}
\usepackage[utf8]{inputenc}
\usepackage[english]{babel}

\usepackage{lipsum}
\newenvironment{Proof}[1][Proof]
  {\proof[#1]\leftskip=1cm\rightskip=1cm}
  {\endproof}
\newcommand{\floor}[1]{\lfloor #1 \rfloor}
\DeclareMathOperator*{\argmax}{arg\,max}
\DeclareMathOperator*{\argmin}{arg\,min}
\makenomenclature
\graphicspath{ {./images/} }
\allowdisplaybreaks
\DeclareMathOperator*{\plim}{plim}
\allowdisplaybreaks


\pagestyle{plain}

%% This bit allows you to either specify only the files which you wish to
%% process, or `all' to process all files which you \include.
%% Krishna Sethuraman (1990).
%
%\typein [\files]{Enter file names to process, (chap1,chap2 ...), or `all' to
%process all files:}
%\def\all{all}
%\ifx\files\all \typeout{Including all files.} \else \typeout{Including only \files.} \includeonly{\files} \fi

\setcounter{tocdepth}{5}
\setcounter{secnumdepth}{5}


\begin{document}
\include{cover}
% Some departments (e.g. 5) require an additional signature page.  See
% signature.tex for more information and uncomment the following line if
% applicable.
% \include{signature}
\pagestyle{plain}

%% This is an example first chapter.  You should put chapter/appendix that you
%% write into a separate file, and add a line \include{yourfilename} to
%% main.tex, where `yourfilename.tex' is the name of the chapter/appendix file.
%% You can process specific files by typing their names in at the 
%% \files=
%% prompt when you run the file main.tex through LaTeX.
\chapter{Computational Components and Execution}
%\begin{table}[!h]
\caption{ARIMA Results - Numeric}
 \label{ARIMA Results - Numeric}
\captionsetup{justification=centering} 
 \begin{center}\begin{tabular}{|c|c|}\hline
{} &      AMZN \\ \hline
Testing, ARIMA and Regression, Numeric  &  0.000346 \\ \hline
Testing, ARIMA, Numeric                 &  0.000337 \\ \hline
Training, ARIMA and Regression, Numeric &  0.000056 \\ \hline
Training, ARIMA, Numeric                &  0.000056 \\ \hline
\end{tabular}
 \end{center}
\end{table}
%\begin{table}[!h]
\caption{ARIMA Results - Numeric}
 \label{ARIMA Results - Numeric}
\captionsetup{justification=centering} 
 \begin{center}\begin{tabular}{|c|c|}\hline
{} &      AMZN \\ \hline
Testing, ARIMA and Regression, Numeric and Text  &  0.000641 \\ \hline
Testing, ARIMA, Numeric and Text                 &  0.000337 \\ \hline
Training, ARIMA and Regression, Numeric and Text &  0.000026 \\ \hline
Training, ARIMA, Numeric and Text                &  0.000056 \\ \hline
\end{tabular}
 \end{center}
\end{table}
\begin{table}[!h]
\caption{ARIMA MSE - Numeric and Text}
 \label{ARIMA MSE - Numeric and Text}
\captionsetup{justification=centering} 
 \begin{center}\begin{tabular}{|c|c|}\hline
{} &      AMZN \\ \hline
Training, ARIMA, Numeric                         &  0.000056 \\ \hline
Training, ARIMA and Regression, Numeric          &  0.000056 \\ \hline
Training, ARIMA and Regression, Numeric and Text &  0.000026 \\ \hline
Testing, ARIMA, Numeric                          &  0.000337 \\ \hline
Testing, ARIMA and Regression, Numeric           &  0.000346 \\ \hline
Testing, ARIMA and Regression, Numeric and Text  &  0.000641 \\ \hline
\end{tabular}
 \end{center}
\end{table}

\begin{table}[!h]
\caption{Linear Regression Results - Percent Change Using Textual Features}
 \label{Linear Regression Results - Percent Change Using Textual Features}
\captionsetup{justification=centering} 
 \begin{center}\begin{tabular}{|c|c|}\hline
{} &       AMZN \\ \hline
Percent Change in Training Adjusted R2 &  7450.4878 \\ \hline
Percent Change in Testing R2           &    -2.1446 \\ \hline
Percent Change in Training MSE         &    -5.2512 \\ \hline
Percent Change in Testing MSE          &     9.6289 \\ \hline
\end{tabular}
 \end{center}
\end{table} % tables

\end{document}

